\documentclass[namedreferences]{solarphysics}
\usepackage[optionalrh]{spr-sola-addons} % For Solar Physics 
%\usepackage{epsfig}          % For eps figures, old commands
\usepackage{graphicx}        % For eps figures, newer & more powerfull
%\usepackage{courier}         % Change the \texttt command to courier style
\usepackage{natbib}         % For citations: redefine \cite commands
%\usepackage{amssymb}        % useful mathematical symbols
\usepackage{color}           % For color text: \color command
\usepackage{url}             % For breaking URLs easily trough lines
\def\UrlFont{\sf}            % define the fonts for the URLs

\usepackage{txfonts}
\usepackage{verbatim}
\usepackage{url}
\usepackage{lscape}
\usepackage{floatrow}

% General definitions
% please place your own definitions here and don't use \def but
% \newcommand{}{} or 
% \renewcommand{}{} if it is already defined in LaTeX

\newcommand{\BibTeX}{\textsc{Bib}\TeX}
\newcommand{\etal}{{\it et al.}}

% Definitions for equations
\renewcommand{\vec}[1]{{\mathbfit #1}}
\newcommand{\deriv}[2]{\frac{{\mathrm d} #1}{{\mathrm d} #2}}
\newcommand{\rmd}{ {\ \mathrm d} }
\newcommand{\uvec}[1]{ \hat{\mathbf #1} }
\newcommand{\pder}[2]{ \f{\partial #1}{\partial #2} }
\newcommand{\grad}{ {\bf \nabla } }
\newcommand{\curl}{ {\bf \nabla} \times}
\newcommand{\vol}{ {\mathcal V} }
\newcommand{\bndry}{ {\mathcal S} }
\newcommand{\dv}{~{\mathrm d}^3 x}
\newcommand{\da}{~{\mathrm d}^2 x}
\newcommand{\dl}{~{\mathrm d} l}
\newcommand{\dt}{~{\mathrm d}t}
\newcommand{\intv}{\int_{\vol}^{}}
\newcommand{\inta}{\int_{\bndry}^{}}
\newcommand{\avec}{ \vec A}
\newcommand{\ap}{ \vec A_p}

\newcommand{\bb}{\vec B}
\newcommand{\jj}{ \vec j}
\newcommand{\rr}{ \vec r}
\newcommand{\xx}{ \vec x}

% Definitions for the journal names
\newcommand{\adv}{    {\it Adv. Space Res.}} 
\newcommand{\annG}{   {\it Ann. Geophys.}} 
\newcommand{\aap}{    {\it Astron. Astrophys.}}
\newcommand{\aaps}{   {\it Astron. Astrophys. Suppl.}}
\newcommand{\aapr}{   {\it Astron. Astrophys. Rev.}}
\newcommand{\ag}{     {\it Ann. Geophys.}}
\newcommand{\aj}{     {\it Astron. J.}} 
\newcommand{\apj}{    {\it Astrophys. J.}}
\newcommand{\apjl}{   {\it Astrophys. J. Lett.}}
\newcommand{\apss}{   {\it Astrophys. Space Sci.}} 
\newcommand{\cjaa}{   {\it Chin. J. Astron. Astrophys.}} 
\newcommand{\gafd}{   {\it Geophys. Astrophys. Fluid Dyn.}}
\newcommand{\grl}{    {\it Geophys. Res. Lett.}}
\newcommand{\ijga}{   {\it Int. J. Geomagn. Aeron.}}
\newcommand{\jastp}{  {\it J. Atmos. Solar-Terr. Phys.}} 
\newcommand{\jgr}{    {\it J. Geophys. Res.}}
\newcommand{\mnras}{  {\it Mon. Not. Roy. Astron. Soc.}}
\newcommand{\nat}{    {\it Nature}}
\newcommand{\pasp}{   {\it Pub. Astron. Soc. Pac.}}
\newcommand{\pasj}{   {\it Pub. Astron. Soc. Japan}}
\newcommand{\pre}{    {\it Phys. Rev. E}}
\newcommand{\solphys}{{\it Solar Phys.}}
\newcommand{\sovast}{ {\it Soviet  Astron.}} 
\newcommand{\ssr}{    {\it Space Sci. Rev.}} 


%%%%%%%%%%%%%%%%%%%%%%%%%%%%%%%%%%%%%%%%%%%%%%%%%%%%%%%%%%%%%%%%%%
\begin{document}

\begin{article}

\begin{opening}

\title{Solar Active Regions and the Global Magnetic Cycle}

\author{P.\,A.~\surname{Higgins}$^{1,2}$\sep
        D.\,S.~\surname{Bloomfield}$^{1}$\sep
        P.\,T.~\surname{Gallagher}$^{1}$      
       }
\runningauthor{Higgins et al.}
\runningtitle{Solar Active Regions and the Global Magnetic Cycle}

   \institute{$^{1}$ Astrophysics Research Group, School of Physics, Trinity College Dublin, Dublin 2, Ireland
                     email: \url{pohuigin@gmail.com}\\ 
              $^{2}$ Lockheed Martin Solar and Astrophysics Laboratory, Palo Alto, CA, USA
             }

\begin{abstract}
Currently, the most successful model of the mechanisms driving the solar magnetic field reversal at the poles is the Babcock-Leighton-Moser model, which is able to qualitatively reproduce the large-scale solar magnetic field over time by simulating the surface transport of injected active regions (ARs). The model relies on empirical knowledge of AR properties, surface flows (including Moser's meridional flow), and the response of surface magnetic fields to those flows. To better constrain the model, accurate measurements of solar magnetic fields over the solar cycle, such as the statistical properties of ARs, are required. This work characterises the long-term patterns of AR emergence and evolution that are responsible for the progression of the observed magnetic solar cycle. The SolarMonitor Active Region Tracker (SMART) is used to automatically detect and characterise solar ARs using 15\,years of \emph{SOHO}/MDI line-of-sight magnetograms covering solar cycle 23. The heliographic location, area, magnetic flux, and flux imbalance of ARs are measured and compared to the results of a global magnetic flux assimilation and potential field spherical harmonic decomposition model. Several novel results are highlighted in this work. Three periods of increased average AR flux and area appear to be largely responsible for the global polar magnetic field reversal. Surface flux transport (the differential and meridional flows and supergranular dispersal) preferentially drag one polarity of flux from ARs poleward, as explained by the Babcock-Leighton-Moser model. For the first time, the magnetic properties of ARs are quantitatively compared to the global configuration of the solar magnetic field; the observed latitudinal bands of imbalanced flux (both dispersed and contained in ARs) determine the magnetic structure of the lower corona to a great extent. We conclude that these observations are consistent with the Babcock-Leighton-Moser model and furthermore, they allow the properties of the magnetic flux injected into simulations as well as the mechanisms that transport the flux to be constrained.
%an excess of ARs with large flux is found in the ``maximum" phase (2000\,--\,2003) around solar maximum
%We observe the following hemispheric asymmetries through the solar cycle using measurements of magnetic flux: i) the double peak during cycle 23 is only seen in the southern hemisphere; ii) the equatorward migration of AR emergence is significantly faster in the northern hemisphere; iii) ARs emerge for longer and more flux emerges overall in the southern hemisphere.
%Distributions of AR flux and area are scale-free, which indicates that the solar dynamo may operate in a self-organised critical state.
%Large-scale flux imbalance in the AR latitudes is opposite but similar in magnitude to that at higher latitudes. This is the first quantitative measurement connecting AR flux with diffuse high-latitude flux. 
%Considering the global magnetic field, similar large-scale features are observed in coronal images and PFSS extrapolations.
%By investigating the spherical harmonic coefficients obtained from the PFSS model, we find that although high-order moments ($l$$=$$[2,3,4,5]$, $m$$=$$0$) become dominant at the solar surface during certain times in the solar cycle, the large-scale magnetic field is shown to become less multipolar with height, tending toward the dipole moment.
%To investigate how the magnetic properties of ARs statistically change over the solar cycle and establish a quantitative connection between their properties and the global magnetic field configuration of the Sun. 


%\verb+SOLA_keyword_list.txt+.  
\end{abstract}
%\keywords{Sun: activity - Sun: magnetic field - Sun: photosphere - Sun: sunspots - Sun: surface magnetism - Sun: dynamo -  Sun: evolution}
%\keywords{Flares, Dynamics; Helicity, Magnetic; Magnetic fields, Corona}
\end{opening}
%-------------------------------------------------

\section{Introduction}

%P-intro Paragraph
Babcock-Leighton-Mosher (BLM) model \citep{Babcock:1961,Leighton:1964,Mosher:1977} explains the global solar magnetic dipole reversal by considering the properties of emergent active regions (ARs) and the velocity field that characterises the solar surface. Simulations built on these principles can approximate the evolution of the large-scale solar magnetic field \citep{Leighton:1964,Sheeley:1985,Devore:1986,Wang:1989}. For instance, a significant number of studies have specifically focused on predicting the evolution of the polar magnetic fields using this model and injecting simulated or observed ARs \citep{Devore:1987,schussler:2006,Wang:2009,Schrijver:2008b,hathaway}. Since the BLM model must be seeded using ARs (simulated or observed), knowledge of their properties is essential. Despite this, there have been few investigations determining the statistical magnetic properties of ARs at different phases of the solar cycle \citep{meunier:2003,zharkov:2006}, knowledge of which would be necessary to make the BLM model both self consistent and predictive. Also, it is known that ARs are important in determining the global magnetic field configuration\citep{wang:2003a, Schrijver:2003, schussler:2006}, but the authors are not aware of work directly comparing the properties of ARs to the global magnetic field. By studying AR magnetic fields and the diffuse fields surrounding them, in this work we diagnose the subsurface dynamo that forms ARs and determine how they evolve, once emerged, to affect the large-scale solar magnetic field configuration. 
%using automated feature detection over a complete solar cycle, 
%using a global assimilation and spherical harmonic decomposition model.

%P-what are sunspots/ARs/MFs -%%%%%%%%%%%%%%%%%%%%%%%%%%%%%%%%%%
There is significant diversity in the properties of localised magnetic features observed over the solar surface. The term AR is used here to denote an entire localised structure, composed of plage and/or sunspots, while the term ``magnetic features" (MFs) is used in this work to denote any regions of strong magnetic field, including ARs. ARs usually emerge as compact, magnetically bipolar features, but both cumulatively \citep{meunier:2003} and at a given point in time can range in size and flux by orders of magnitude \citep{Parnell:2009}. The statistical properties of observed ARs provide limited insight into the mechanisms responsible for their emergence if they are considered with respect to time, since the properties of emergent ARs may change over the solar cycle.

%Sunspots are localised regions of strong magnetic field that appear darker and cooler than the surrounding photospheric continuum.
%Sunspots generally occur in clusters called sunspot groups and may be surrounded by regions of dispersing magnetic fields called ``plage".

%P-The solar cycle -%%%%%%%%%%%%%%%%%%%%%%%%%%%%%%%%%%%%%%%%%
The solar cycle is conventionally characterised by the time-dependent statistical properties of sunspot groups, which exhibit roughly 11- and 22-year periodicity. The number of sunspot groups present \citep[sunspot number][]{Schwabe:1844,Wolf:1861}, their latitudinal dependence \citep[Sp$\ddot{\mbox{o}}$rer's law][]{Maunder:1904), and the inclination of their bipole axis to the equator \cite[Joy's Law][]{Hale:1919}, which decreases with latitude \citep{Howard:1991}, exhibit an 11-year periodicity. Meanwhile, the magnetic polarity of leading and following sunspot groups (hale's law; hale 1919) follows a 22-year periodicity. Additionally, sunspot groups appear to exhibit a similar size distribution over a solar cycle\citep{harvey:1993}, but may show an excess of large regions or a depletion of small regions at cycle maximum \citep{tang:1984,Hathaway:2010b,Lefevre:2011,Kilcik:2011,deToma:2013}. The statistical properties of sunspot groups over time are well studied using white-light and continuum images, but the cycle dependence of detected AR size, flux, and net flux using magnetogram data have not been extensively studied \citep{meunier:2003,zharkov:2006}. 

%P-how ARs drive the magnetic solar cycle -%%%%%%%%%%%%%%%%%%%%%%%%%%%%%
The evolution of ARs, the specifics of which are modulated by their properties over the solar cycle, drives the magnetic polarity reversal of the global magnetic field. 
ARs evolve due to the turbulent motions and large-scale flows of the near-surface solar plasma: differential axial-rotation of the Sun stretches features longitudinally \citep{Babcock:1961}; supergranule motions result in magnetic field dispersal \citep{Leighton:1964}, submergence, and cancellation; the meridional flow preferentially carries diffuse magnetic flux to the poles \cite{Mosher:1977}. The changing properties of ARs over the solar cycle work in concert with these phenomena to drive the reversal of the polar magnetic fields around the peak of each solar cycle. While this is well known, the quantitative result of these effects on the magnetic properties of ARs has not been well studied.

%P-the magnetic solar cycle/ARs determine the global field structure -%%%%%%%%%%%%%%%%%
It has been determined that to accurately predict the properties of the magnetic field of the Sun at large radii, the properties of low-latitude MFs on the solar surface must be taken into account \citep{wang:2003a, Schrijver:2003, schussler:2006}. The distribution of open flux and the global field configuration determine the shape of the heliospheric field, and thus can affect the space weather environment at Earth.
Here, we seek to better understand the evolution of the global solar magnetic field in the context of a full 11-year activity cycle (i.e., half a magnetic cycle) by characterizing spatio-temporal distributions of MF properties.

Generally, the global field is multipolar and we can describe its 3D geometry using the superposition of spherical harmonics from surface magnetic field measurements \citep{stenflo:1986, stenflo:1988, knaack:2005,mordvinov:2007,Mackay:2012,DeRosa:2012}.

Properties of individual ARs (such as flux imbalance) locally affect the solar magnetic field structure, while large-scale properties (such as hemispheric flux imbalance) affect the global field.

Observations of the latitudinal distribution of net flux over time show that while the AR belt progresses from high to low latitudes, unipolar flux is observed to be transported poleward. This has been observed in magnetic butterfly diagrams \citep{harvey:1992}, but has not been quantified.
\citet{Choudhary:2002} investigated large-scale flux imbalance, finding that hemispheric flux imbalance is a function of the solar cycle and switches sign between cycles. \citet{zharkov:2006} statistically studied the excess flux of sunspot groups  for a portion of solar cycle 23, finding opposite imbalances in each hemisphere that reverse at the cycle minimum. \citet{Zharkov:2008} state that a phase relation exists between AR flux imbalance and the background high-latitude magnetic field. 


%P-overview of the paper -%%%%%%%%%%%%%%%%%%%%%%%%%%%%%%%%%%%%%%
In this paper, which is based on work presented in Higgins (2013), we analyse the properties of detected ARs over solar cycle 23 and compare them to the global configuration of the solar magnetic field. %(Section \ref{subsect_imbharm}). 
We establish a clear connection between strong-field features present in the AR belt, the excess of weak high-latitude net flux, and the global magnetic field structure that is determined by the dominance of certain spherical harmonics. The complete life cycle of surface magnetic flux has not been quantitatively studied using both automated feature detection and global field measurements.  
In Section~\ref{obs_and_meth}, we discuss the methods used to construct our data set including the automatic detection algorithm, data reduction processes and the assimilation and spherical harmonic decomposition model used. Our results are presented and discussed in Section~\ref{results}, and concluding statements are made in Section~\ref{discussion}.











 
\section{General Text} %%%%%%%%%%%%%%%%%%%%%%%%%%%%%%%%%%%%%%%%
      \label{S-general}      

\subsection{Text with Citations} %%%%%%%%%%%%%%
  \label{S-text}
This section gives an example of text with references 
included with the \verb+\cite{}+ and \verb+\opencite{}+ commands
(see Section~\ref{S-references} for more  commands).

   %{\S}{\bf --- Introduction of H.  What is it ?} \\
Magnetic helicity quantifies how the magnetic field is sheared
and/or twisted compared to its lowest energy state (potential
field). Observations of sheared, and even helical, magnetic
structures in the photosphere, corona and solar wind have
attracted considerable attention, with the consequent interest in
magnetic helicity studies (see reviews by \opencite{Brown99}, and,
\opencite{Berger03}). Stressed magnetic fields are often observed
in association with flares, eruptive filaments, and coronal mass
ejections (CMEs), but the precise role of magnetic helicity in
such activity events still needs to be clarified.

   %{\S}{\bf --- Theoretical use of H} \\
Magnetic helicity plays a key role in magnetohydrodynamics (MHD)
because it is almost preserved on a timescale less than the global
diffusion timescale (\opencite{Berger84}, \citeyear{Berger03}).  
Its conservation defines a
constraint on the magnetic field evolution; in particular a
stressed magnetic field with finite total helicity cannot relax to
a potential field.  Thus magnetic helicity is at the heart of
several MHD relaxation theories, for example of coronal heating
\cite{Heyvaerts84} but also of flares \cite{Kusano04,Melrose04}.
The permanent accumulation of helicity in the corona could be
vital to the origin of CMEs \cite{Rust94,Low97}.  In the
convection zone, the accumulation of helicity in large scales
limits the efficiency of the dynamo, thus the conservation of
magnetic helicity is responsible for dynamo saturation, the
so-called $\alpha$-effect quenching \cite{Brandenburg01}.

\subsection{Importance of Using Labels} %%%%%%%%%%%%%%
  \label{S-labels}

  %{\S}{\bf --- Why using labels ?} \\
    \LaTeX\ defines labels for many features like sections, 
equations, figures, tables, and citations.  The systematic use
of these labels greatly facilitates the writing of a scientific
article (even if it may appear as more extra work at the beginning).
Indeed, it permits one to re-number or re-order automatically the
features during the compilation ({\it e.g.} when adding or moving a 
section).  It also permits one to
cross check automatically whether the citations have been included
in the bibliography list.

  %{\S}{\bf --- How to use labels} \\
    Labels are powerful but their use can be cumbersome if some
clear logic is not used in defining them since one can easily
forget the exact defined label ({\it e.g.} case sensitive).
A label should be simple
while reflecting precisely what it refers to. It is very helpful
to create a small auxiliary file where all these labels are
kept (see the \verb+SOLA_example_labels.tex+ 
accompanying the present file).   
It also provides a roadmap of the paper with the list
of the sections and subsections with the equations introduced in each.     
Including the full command ({\it e.g.} \verb+Section~\ref{S-labels}+)
permits one to do a simple copy/paste when needed (rather than moving
through the \texttt{.tex} file looking for the definition of the label). 
It is also useful that  
\verb+SOLA_example_labels.tex+   file contains the copy of the 
new commands, as well as the citation commands. For references,
the simple convention of concatenating the first author's name and the year
(and eventually a letter), is simple enough to be easily remembered.
    
\section{Including Special Features} %%%%%%%%%%%%%%%%%%%%%%%%%%%%%%%%%%%%%%%%
      \label{S-features}      

\subsection{Examples of Equations} %%%%%%%%%%%%%%
  \label{S-equations}
Here are a few examples of equations. It is useful to define
a new command when a combination of symbols is present at several 
locations, for example:\\
  \verb+ \renewcommand{\vec}[1]{{\mathbfit #1}} + 
(see the beginning of present \texttt{.tex} file for more examples). 
The mathematics style is to set operators such as ``d'', ``ln'', ``log'', 
``curl'', \textit{etc.} in roman, not italic.

  
\subsubsection{Simple Equations} %%%%%%%%%%%%%%
  \label{S-simple-equations}
 The magnetic helicity of the magnetic field ($\vec{B}$) fully contained 
within a volume $\vol $ is \cite{Elsasser56}:
   \begin{equation}  \label{Eq-H-def}
     H^{\rm closed} = \intv \avec \cdot \bb \dv \,.
   \end{equation}

\subsubsection{Array of Equations} %%%%%%%%%%%%%%
  \label{S-array-equations}
The vector potential ($\avec$) can be written as a function of $\bb$
within the Coulomb gauge:
   \begin{eqnarray}   
     \avec (\xx) &=& \mu_{0}\intv \frac{\jj (\xx  ^\prime)}
                     { | \xx - \xx  ^\prime |} \dv ^\prime          
                       \nonumber \\      % \\ indicates the equation end
                 &=& \frac{1}{4\pi} \int \bb (\xx ^\prime ) \times
                     \frac{(\xx  - \xx ^\prime)}{\,\, |\xx - \xx ^\prime |^3}  
                                           \dv ^\prime   \,. 
                       \label{Eq-A-B}    % No ``\\'' for the last equation
   \end{eqnarray}

Then the magnetic helicity can be written as a function of $\bb$
alone \cite{Moffatt69}. An approximation of this double integral 
can be realized by splitting
the magnetic field in $N$ flux tubes \cite{BergerF84}:
   \begin{eqnarray}   
    H^{\rm closed} &=& \frac{1}{4\pi} \intv \intv
                      \bb (\xx ) \times \bb (\xx ^\prime )
                      \cdot \frac{(\xx  - \xx ^\prime)}{\,\, |\xx - \xx ^\prime |^3}
                      \dv \dv ^\prime \,,                     \label{Eq-H-B} \\      
                  &\approx & \sum_{i=1}^N T_i^{\rm closed} \Phi_i^2
                           + \sum_{i=1}^N \sum_{j=1,j \neq i}^N
                             {\mathcal L}^{\rm closed}_{i,j} \Phi_i \Phi_j  \,.
                       \label{Eq-H-N}     % No ``\\'' for the last equation
   \end{eqnarray}
where $\Phi_i$ and $T_i^{\rm closed}$ are the magnetic flux and the
self helicity of flux tube $i$ respectively ($T_i^{\rm closed}$
includes both twist and writhe), and ${\mathcal L}^{\rm
closed}_{i,j}$ is the mutual helicity between flux tubes $i$ and
$j$.

\subsubsection{Long Equations} %%%%%%%%%%%%%%
  \label{S-long-equations}
 A long equation is broken into several lines:
      \begin{eqnarray}
      \deriv{H}{t} &=& 
                    \frac{1}{2 \pi} \int_{\Phi } \int_{\Phi }
                    \left(   \deriv{ \theta (\xx _{c_-} -\xx _{a_+}) }{t}
                           + \deriv{ \theta (\xx _{c_+} -\xx _{a_-}) }{t}
                    \right.                            \nonumber  \\
       && \qquad \qquad \;
                    \left. - \deriv{ \theta (\xx _{c_+} -\xx _{a_+}) }{t}
                         -\, \deriv{ \theta (\xx _{c_-} -\xx _{a_-}) }{t}
                    \right) 
                    \rmd \Phi_{a} \rmd \Phi_{c}  \,. \label{Eq-dH-Phi}
      \end{eqnarray}
A fine tuning of the positions can be obtained with the following spacing commands
(\verb+\!+ is a negative thin space):
     \begin{eqnarray} 
     \verb+\! + |\!| \,\  \qquad &    \! \verb+  \, + |\,|    
                          & \qquad    \verb+    \: + |\:|       \nonumber \\
     \verb+\ + |\ |       \qquad &    \verb+\quad + |\quad| 
                          & \qquad    \verb+\qquad + |\qquad|   \nonumber  
     \end{eqnarray}


  \begin{figure}    %%%%%%%%%%%%%%%%%% FIGURE 1 
   \centerline{\includegraphics[width=0.5\textwidth,clip=]{fig1a.eps}
              }
              \caption{Example of a simple figure with only one panel. 
Relative units (here \texttt{$\backslash$textwidth}) are preferred
so that the figure adapts automatically to the text width (this command is very useful
in more complex figures such as Figures~\ref{F-4panels} and~\ref{F-rotate-cut}).
The use of the command \texttt{$\backslash$includegraphics} requires the 
inclusion of \texttt{$\backslash$usepackage\{graphicx\}} at the beginning of the
\LaTeX\ file.
                      }
   \label{F-simple}
   \end{figure}

  \begin{figure}    %%%%%%%%%%%%%%%%%% FIGURE 2
                                % includes the two top panels 
   \centerline{\hspace*{0.015\textwidth}
               \includegraphics[width=0.515\textwidth,clip=]{fig1a.eps}
               \hspace*{-0.03\textwidth}
               \includegraphics[width=0.515\textwidth,clip=]{fig1b.eps}
              }
     \vspace{-0.35\textwidth}   % Shift close to the panel top 
     \centerline{\Large \bf     % Includes the labels (here needs the color 
                                %   package, see beginning of this file)
      \hspace{0.0 \textwidth}  \color{white}{(a)}
      \hspace{0.415\textwidth}  \color{white}{(b)}
         \hfill}
     \vspace{0.31\textwidth}    % Shift back to the panel bottom 
%           
   \centerline{\hspace*{0.015\textwidth}
               \includegraphics[width=0.515\textwidth,clip=]{fig1c.eps}
               \hspace*{-0.03\textwidth}
               \includegraphics[width=0.515\textwidth,clip=]{fig1d.eps}
              }
     \vspace{-0.35\textwidth}   % Shift close to the panel top 
     \centerline{\Large \bf     % Includes the labels (here needs the color package)
      \hspace{0.0 \textwidth} \color{white}{(c)}
      \hspace{0.415\textwidth}  \color{white}{(d)}
         \hfill}
     \vspace{0.31\textwidth}    % Shift back to the panel bottom 
              
\caption{Example of a figure with four panels (constructed with 
four \texttt{.eps} files).  
The labels of the panels are included with \LaTeX\ commands 
so that each panel can be referred to unambiguously in the text.
The position of the panels
is fine-tuned with the \texttt{$\backslash$hspace} and 
\texttt{$\backslash$vspace} commands.
        }
   \label{F-4panels}
   \end{figure}

%% The following show the use of the epsfig package.   
%%    Un-comment the relevant parts if needed
%%    Un-comment the \usepackage{epsfig} line at the top of this file
%  \begin{figure}    %%%%%%%%%%%%%%%%%% FIGURE 2 bis
%   \centerline{\epsfig{file=fig1a.eps,width=0.51\textwidth,clip=}
%               \hspace*{-0.03\textwidth}
%               \epsfig{file=fig1b.eps,width=0.51\textwidth,clip=}
%              }
%    \vspace*{0.003\textwidth}
%   \centerline{\epsfig{file=fig1c.eps,width=0.51\textwidth,clip=}
%               \hspace*{-0.03\textwidth}
%               \epsfig{file=fig1d.eps,width=0.51\textwidth,clip=}
%              }
%  \centerline{\includegraphics[width=\textwidth,clip=]{fig1.eps} }                 
%\caption{Example of figure with four panels.}
%   \label{F-4panels-bis}
%   \end{figure}

  \begin{figure}    %%%%%%%%%%%%%%%%%% FIGURE 3   rotation + cut
% Original BoundingBox (in the .eps files) : 54 360 558 720
%   corresponds to the coordinates of: left, bottom, right, top   (unrotated)
   \centerline{\includegraphics[width=0.4\textwidth,clip=,
                 bb=54 440 488 660,angle=-90]{fig1a.eps}
               \includegraphics[width=0.4\textwidth,clip=,
                 bb=54 440 488 660,angle=-90]{fig1b.eps}
               \includegraphics[width=0.4\textwidth,clip=,
                 bb=54 440 488 660,angle=-90]{fig1c.eps}
               \includegraphics[width=0.4\textwidth,clip=,
                 bb=54 440 488 660,angle=-90]{fig1d.eps}
              }
     \vspace{-0.39\textwidth}   % Shift close to the panel top 
     \centerline{\Large \bf     % Includes the labels (here needs the color package)
      \hspace{0.07\textwidth}   \color{white}{(a)}
      \hspace{0.122\textwidth}  \color{black}{(b)}
      \hspace{0.122\textwidth}  \color{white}{(c)}
      \hspace{0.122\textwidth}  \color{white}{(d)}
         \hfill}
     \vspace{0.36\textwidth}    % Shift back to the panel bottom 
\caption{Example of a figure with panels smaller
than the original and rotated clockwise by 90$^{\circ}$ 
(compare with Figure~\ref{F-4panels}). The \texttt{clip=} 
command is important to include only the selected part of the figure 
by changing the \texttt{BoundingBox}. 
The labels of the panels are included using \LaTeX\ commands. 
        }
   \label{F-rotate-cut}
   \end{figure}
   

\subsection{Examples of Figures} %%%%%%%%%%%%%%
  \label{S-figures}
  %{\S}{\bf --- Main features} \\
 A simple figure is presented as Figure~\ref{F-simple}. When more
than one panel is present, one should add labels for those individual panels.
One can add labels to a figure by using \LaTeX\ as done
in Figures~\ref{F-4panels} and~\ref{F-rotate-cut}. 
The package \verb+\usepackage{color}+
can be used to write text ({\it e.g.} labels) in white or in color. 
Figures can be rotated and their position fine tuned 
(Figures~\ref{F-4panels} and~\ref{F-rotate-cut}).

  %{\S}{\bf --- Cutting a figure in the .eps file} \\
  Cutting a figure can be made by editing the \texttt{.eps} 
file with a text editor, and changing the \texttt{BoundingBox}, 
then saving the file (do not use a text editor designed for \LaTeX\ since 
it could open the file as a figure, and not as a text file).
An \texttt{.eps} file has typically the command
\verb+ %%BoundingBox: 54 360 558 720 + 
at the beginning, where the numbers are the left, bottom, right, and top
coordinates of the graphic (in units of ``pt'').  
Changing these numbers is a way to reduce the part of the image shown. 
The \texttt{GhostView} application gives the coordinates
of the cursor (in units of ``pt''), so it permits one to locate the 
coordinates of the cropping. 
The result of the changes can be checked using \texttt{GhostView}. 
 Note that, depending on the software used to create the \texttt{.eps} file, 
the \texttt{BoundingBox} can be repeated at several places in the
\texttt{.eps} file ({\it e.g.} with \texttt{PageBoundingBox}). 
Also, with some software,
the \texttt{BoundingBox} is defined only close to the end of the file
(the file has at the beginning: \texttt{BoundingBox: (atend)}).
The \texttt{BoundingBox} can still be changed
in place, or defined at the beginning of the \texttt{.eps} file. 
   Finally, this method provides a figure with a reduced size, when
included in \LaTeX\ (do not forget the \texttt{clip=} 
in the command including the \texttt{.eps} file!).     
The advantage of this method is that the correct \texttt{BoundingBox}
is easily determined.

  %{\S}{\bf --- Cutting a figure with \LaTeX } \\
   An alternative way to crop figures is to include the \texttt{BoundingBox}
in the\linebreak \verb+\includegraphics+ command, for example:\\
\verb+ \includegraphics[width=\textwidth,bb=54 440 488 660, clip=]+ \\
  The advantage
is that it can be made within \LaTeX . The initial value is given
by the \texttt{BoundingBox} found in the \texttt{.eps} file.
It may be better to process the
figure in a separate \texttt{.tex} file, since it will require several 
iterations to get the right \texttt{BoundingBox}.
   
   

\begin{table}
\caption{ A simple table. Each column is aligned by one of the letters:
l: left, c: center, r: right. 
Using two \$s permits one to insert equation-like features (see last column). 
The inclusion of $\sim$ adds a blank to approximately align the numbers
of the last two columns (see the \LaTeX\ file).
}
\label{T-simple}
\begin{tabular}{ccclc}     % define the column alignment
                           % l: left, c: center, r: right
  \hline                   % horizontal line
Rot. & Date & CMEs & CMEs~ & $\alpha$ \\
     &      & obs. & ~cor. & $10^{-2}$Mm$^{-1}$\\
  \hline
1\tabnote{First table line.} & 02--Nov--97 & 16  & 24.1~ & -1.26 \\
2 & 29--Nov--97 & --  & ~2.53 & ~0.94 \\
3 & 27--Dec--97 & 06  & 11.7~ & ~0.82 \\
4 & 23--Jan--98 & 09  & 16.82 & ~0.94 \\
5 & 20--Feb--98 & 04  & ~9.6~ & ~1.00 \\
total&          & 35  & 64.75 &       \\
  \hline
\end{tabular}
\end{table}
  
  \begin{table}
\caption{ A more complex table with multi-columns labels. The command 
\texttt{$\backslash$multicolumn\{4\}\{c\}\{Flares (GOES)\}} permits
writing the title ``Flares (GOES)'' over four columns. The alignment
of the decimal points is made by defining two columns separated with an
inter-column replaced by a ``.'' with the command \texttt{r@\{.\}l}  
(see the \LaTeX\ file).
}
\label{T-complex}
\begin{tabular}{lcccccc r@{.}l c} % define the column alignment
                                  % l: left, c: center, r: right
                                  % @{.} replace the inter-column by a .
  \hline
Rot. & Date & \multicolumn{4}{c}{Flares (GOES)}& CMEs 
     & \multicolumn{2}{c}{CMEs} & $\alpha$ \\
     &      &   X & M & C & B                  & obs. 
     & \multicolumn{2}{c}{cor.} & $10^{-2}$Mm$^{-1}$\\
  \hline
1 & 02--Nov--97 & 02 & 04 & 24 & 05 & 16  & ~24&1 & -1.26 \\
2 & 29--Nov--97 & -- & -- & 03 & 04 & --  &   2&53& ~0.94 \\
3 & 27--Dec--97 & -- & 01 & 07 & 08 & 06  &  11&7 & ~0.82 \\
4 & 23--Jan--98 & -- & -- & 03 & 03 & 09  &  16&82& ~0.94 \\
5 & 20--Feb--98 & -- & -- & -- & -- & 04  &   9&6 & ~1.00 \\
total&          & 02 & 05 & 37 & 20 & 35  &  64&75&       \\
  \hline
\end{tabular}
\end{table}

\subsection{Examples of Tables} %%%%%%%%%%%%%%
  \label{S-tables}
   Tables are easy to write provided one keeps the alignment 
with the column separator \texttt{\&} when entering the table 
in the \LaTeX\ file 
(even if it is not required by \LaTeX). Examples of a simple table,
Table~\ref{T-simple}, and a more complex table, Table~\ref{T-complex},
are given.

\subsection{Including References in the Text} %%%%%%%%%%%%%%
  \label{S-references}
  The classical way to input references in the text is with the 
\verb+\cite{label-ref}+ command where \verb+label-ref+ is a label
unique for each reference.  It is defined in the environment
\verb+\begin{thebibliography}{} ... +, or in the \BibTeX\ file
(see Section~\ref{S-BibTeX}). The main citation commands, and their
compilation results, are:\\
\verb+\cite{Kusano04}          +: (Kusano \etal, 2004)\\
\verb+\cite{Kusano04,Berger03} +: (Kusano \etal, 2004; Berger, 2003)\\
\verb+\inlinecite{Brown99}     +: Brown, Canfield, and Pevtsov (1999)\\
\verb+\opencite{Brown99}       +: Brown, Canfield, and Pevtsov, 1999\\
\verb+\citeauthor{Brown99}     +: Brown, Canfield, and Pevtsov\\
\verb+\shortcite{Brown99}      +: (1999)\\
\verb+\citeyear{Brown99}       +: 1999\\
\verb+(\opencite{Berger84}, \citeyear{Berger03}) +: 
                                  (\opencite{Berger84}, \citeyear{Berger03})
   
\subsection{Using \BibTeX} %%%%%%%%%%%%%%
  \label{S-BibTeX}
  %{\S}{\bf --- Why using \BibTeX ? } \\
  The use of \BibTeX\ simplifies the inclusion of references. Only the 
references cited and labeled in the text are included at compilation, 
and an error message appears if some references
are missing.  Any new reference will automatically be written at the correct 
location in the reference list after compilation. 
Moreover the references are stored, in any order, in a separate file
(with the \texttt{.bib} extension) in the \BibTeX\ format, so independently of 
the journal format. Such a personal reference file can be re-used with any journal.
The formatting of the references and their listing order are made automatically
at compilation (using the information given in the \texttt{.bst} file). 
        
  %{\S}{\bf --- Downloading references} \\
  The references in \BibTeX\ format can be downloaded from the 
Astrophysics Data System (ADS), then stored
in \verb+sola_bibliography_example.bib+  (file name of the present example).
The main extra work is to define a proper and easy label for each citation
(a convenient one is simply first-author-name-year).  Furthermore, it is better
to have the journal names defined by commands (for example 
\texttt{$\backslash$solphys)}, as defined at the beginning of 
this \texttt{.tex} file.
This provides an homogeneity in the reference list and permits flexibility
when changing for journals.   Some caution should be taken for some journals
since ADS does not necessarily provide a uniform format for the
journal names. This is the case for \jgr\  Moreover since
\jgr\ has a new way to refer to an article 
(since 2002 it has no page number), then the ADS references need to be corrected. 
More generally, it is worth verifying
each reference from the original publication (independently of \BibTeX\ use).

  %{\S}{\bf --- Compilation: general} \\
   The full \LaTeX\ and \BibTeX\ compilation is made in four steps: 
\begin{tabbing}
1) {\tt latex filename}\qquad\qquad\=(stores the labels in the {\tt .aux} file)\\
2) {\tt bibtex filename}\>(loads the bibliography in the {\tt .bbl} file)\\
3) {\tt latex filename}\>(reads the .bbl, stores in the {\tt .aux})\\
4) {\tt latex filename}\>(replaces all labels)  
\end{tabbing}
   where \texttt{filename} is the name of your \LaTeX\ file (for example, 
the present file) {\bf without} typing its \texttt{.tex} extension.
If a \texttt{(?)} is still present in the output (at the place of a label),
it means that this label has not been properly defined. 
 (for example, \LaTeX\ labels are case sensitive).
Any undefined label has a warning written in the \texttt{console window}
(it is better to have this window open by default, since \LaTeX\ warning and 
error messages are very useful to localize the problem).

  %{\S}{\bf --- Compilation: simple} \\
  When the references are not changed, it is unnecessary to re-run \BibTeX .
When no new labels are added, running latex once is sufficient to refresh
the \LaTeX\ output. So, except for
the first, and the final time (safest), running \LaTeX\ once is sufficient
in most cases to update the \LaTeX\ output, if the compilation files 
created are not erased! For example \BibTeX\ keeps the bibliography in the usual 
environment,\\
  \verb+ \begin{thebibliography}{} ... \end{thebibliography}+\\
in the file with the \verb+.bbl+ extension.  

\subsection{Miscellaneous Other Features} %%%%%%%%%%%%%%
      \label{S-Miscellaneous} 
Long URL's can be quite messy when broken across lines
\texttt{ http://gong.nso.edu/data/magmap/} as normal text,
however the \texttt{url} package does a nice job of this, \textit{e.g.} 
\url{http://gong.nso.edu/data/magmap/}.
   
\section{Conclusion} %%%%%%%%%%%%%%%%%%%%%%%%%%%%%%%%%%%%%%%%
      \label{S-Conclusion} 
      We hope authors of {\it Solar Physics} will find this guide useful.
Please send us feedback on how to improve it.
      
  \LaTeX\ is very convenient to write a scientific text, in particular
with the use of labels for figures, tables, and references. Moreover, the labels and list of references are checked by the software against
one another, and, the formatting should be effortless with \BibTeX.

%%%%%%%%%%%%%%%%%%%%%%%%%%%%%%%%%%%%%%%%%%%%%%%%%%%%%%%%%%%%%%%%%%%%%%%%%%%
\appendix   

 After the \verb+\appendix+ command, the sections are referenced with 
capital letters. 
The numbering of equations, figures and labels is 
is just the same as with classical sections.

  \begin{figure}    %%%%%%%%%%%%%%%%%% FIGURE 1 
   \centerline{\includegraphics[width=0.3\textwidth,clip=]{fig1a.eps}
              }
   \caption{Example of a simple figure in an appendix.}
    \label{F-appendix}
  \end{figure}

  \begin{table}
   \caption{ A simple table in an appendix. }
   \label{T-appendix}
    \begin{tabular}{ccclc}     % define the column alignment
                               % l: left, c: center, r: right
      \hline                   % horizontal line
    Rot. & Date & CMEs & CMEs~ & $\alpha$ \\
         &      & obs. & ~cor. & $10^{-2}$Mm$^{-1}$\\
      \hline
    1 & 02--Nov--97 & 16  & 24.1  & -1.26 \\
    2 & 29--Nov--97 & --  & ~2.53 & ~0.94 \\
      \hline
    \end{tabular}
   \end{table}


\section{Abbreviations of some Journal Names} %%%%%%%%%
    \label{S-appendix}
Journal names are abbreviated in {\it Solar Physics} with the IAU
convention (IAU Style Book
published in Transactions of the IAU XXB, 1988, pp. Si-S3.
\url{www.iau.org/Abbreviations.235.0.html}).  Here are a few journals with their \LaTeX\ 
commands (see the beginning of this \texttt{.tex} file).\\
  \verb+\aap     + \aap \\
  \verb+\apj     + \apj \\
  \verb+\jgr     + \jgr \\
  \verb+\mnras   + \mnras \\
  \verb+\pasj    + \pasj \\
  \verb+\pasp    + \pasp \\
  \verb+\solphys +~ \solphys 
  
%%%%%%%%%%%%%%%%%%%%%%%%%%%%%%%%%%%%%%%%%%%%%%%%%%%%%%%%%%%%%%%%%%%%%%%%%%%
\begin{acks}
 The authors thank ... ({\it note the reduced point size})
\end{acks}


%%% BIBLIOGRAPHY %%%%%%%%%%%%%%%%%%%%%%%%%%%%%%%%%%%%%%%%%%%%%%%%%%%%%%%%%%%
\mbox{}~\\ 
\noindent {\normalsize \bf Bibliography Included with \BibTeX }\\* 
      % more powerful
  With \BibTeX\ the formatting will be done automatically for all 
the references cited with one
of the \verb+\cite+ commands (Section~\ref{S-references}).
Besides the usual items, it includes the title of the article 
and the concluding page number. 
   
     % format of references provided by the journal (.bst)
\bibliographystyle{spr-mp-sola}
%\bibliographystyle{spr-mp-sola-cnd} %% Alternative style: no title,
                                      % no concluding page. 

     % name your Bibtex file containing your references (.bib)
\bibliography{sola_bibliography_example}  

     % Checking: look if the file containing the ``\bibitem'' exits
     %           so check if the .bbl file exist (bibTeX compilation)
\IfFileExists{\jobname.bbl}{} {\typeout{}
\typeout{****************************************************}
\typeout{****************************************************}
\typeout{** Please run "bibtex \jobname" to obtain} \typeout{**
the bibliography and then re-run LaTeX} \typeout{** twice to fix
the references !}
\typeout{****************************************************}
\typeout{****************************************************}
\typeout{}}

\noindent {\normalsize \bf Bibliography included manually }\\*
     % Require more work
  The articles can be entered, formatted, and ordered  
by the author with the command \verb+\bibitem+.  ADS provides
references in the {\it Solar Physics} format by selecting
the format \verb+SoPh format+ under the menu 
\verb+Select short list format+.    Including the article title
and the concluding page number are optional;
however, we require consistency in the author's choice.
That is, all of the references should have the article title, or none,
and similarly for ending page numbers.

\begin{thebibliography}{}
  \bibitem[\protect\citeauthoryear{{Berger}}{2003}]{Berger03b}
Berger,~M.A.: 
2003, in Ferriz-Mas, A., N{\'u}{\~n}ez, M. (eds.),
    \textit{Advances in Nonlinear Dynamics}, Taylor and Francis Group, 
    London, 345.
  \bibitem[\protect\citeauthoryear{{Berger} and {Field}}{1984}]{BergerF84b}
Berger,~M.A., Field,~G.B.: 
1984, \textit{J. Fluid. Mech.} \textbf{147}, 133.
  \bibitem[\protect\citeauthoryear{{Brown}, {Canfield}, and
                                   {Pevtsov}}{1999}]{Brown99b}
Brown,~M., Canfield,~R., Pevtsov,~A.:
1999, Magnetic Helicity in Space and Laboratory Plasmas, Geophys. Mon. 
      Ser. 111, AGU.
 \bibitem[\protect\citeauthoryear{{Dupont}, {Schmidt}, and {Koutny}}{2007}]{Dupont07b}
Dupont, J.-C., Schmidt, F., Koutny, P.: 2007, \solphys{} \textbf{323}, 965. 
\end{thebibliography}

\end{article} 

\end{document}
